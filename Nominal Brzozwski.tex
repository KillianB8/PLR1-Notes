\documentclass[a4paper, UKenglish, cleveref, autoref, numberwithinsect]{lipics-modified}

\pdfoutput=1
\hideLIPIcs
%\graphicspath{{./assets/}}
\bibliographystyle{unsrt}
\nolinenumbers

\title{Algorithms in Toposes}

\author{Killian {Barbé}}{ENS de Lyon, Lyon, France}{killian.barbe@ens-lyon.fr}{https://orcid.org/0009-0001-8345-9233}{}

\authorrunning{Killian Barbé}
%\usepackage{showframe}

\usepackage{kpfonts}
\renewcommand*\ttdefault{cmvtt}

\usepackage{lipsum}
\usepackage{amsmath}
\usepackage{mathtools}
\usepackage[lined,boxed]{algorithm2e}
\usepackage{bbm}
\usepackage{tabularx}
\usepackage{quiver}
\usepackage{tikz}
\usepackage{adjustbox}
\usepackage{setspace}
\usepackage{silence}

\WarningFilter{microtype}{}

%\setlength{\mathindent}{0pt}
\usetikzlibrary{automata, positioning, arrows, cd, decorations.markings}
\tikzset{negated/.style={
        decoration={markings,
            mark= at position 0.5 with {
                \node[transform shape] (tempnode) {$\slash$};
            }
        },
        postaction={decorate}
    }
}
\allowdisplaybreaks
\newcommand*{\todo}[1][todo]{%
  \protect\leavevmode
  \begingroup
    \color{red}#1%
  \endgroup
}
\renewcommand{\baselinestretch}{1.1} 

\newcommand{\Set}{\mathbf{Set}}
\newcommand{\Cat}{\mathbf{Cat}}
\newcommand{\Rel}{\mathbf{Rel}}
\newcommand{\Auto}{\mathbf{Auto}}
\newcommand{\Nom}{\mathbf{Nom}}
\newcommand{\PSh}{\mathbf{PSh}}

\renewcommand{\geq}{\geqslant}
\renewcommand{\leq}{\leqslant}
\renewcommand{\epsilon}{\varepsilon}
\newcommand{\emp}{\emptyset}
\newcommand{\tp}{\dashv}
\newcommand{\hookto}{\hookrightarrow}

\newcommand{\id}{\mathrm{id}}
\newcommand{\coker}{\mathrm{coker}}
\newcommand{\Eq}{\mathrm{Eq}}
\newcommand{\ob}{\mathrm{ob}}
\newcommand{\op}{\mathrm{op}}
\newcommand{\ev}{\mathrm{ev}}
\newcommand{\Hom}{\mathrm{Hom}}
\newcommand{\dom}{\mathrm{dom}}

\newcommand{\Lan}{\mathsf{Lan}}
\newcommand{\Ran}{\mathsf{Ran}}

\newcommand{\Member}{\mathsf{Member}}
\newcommand{\Equiv}{\mathsf{Equiv}}
\newcommand{\TopL}{\mathsf{Top}L^*}

\newcommand{\row}{\textup{\texttt{row}}}
\newcommand{\pref}{\textup{\texttt{pref}}}
\newcommand{\suff}{\textup{\texttt{suff}}}
\newcommand{\merge}{\textup{\texttt{merge}}}
\newcommand{\rank}{\textup{\texttt{rank}}}

\newcommand{\insf}{\mathsf{in}}
\newcommand{\stsf}{\mathsf{st}}
\newcommand{\outsf}{\mathsf{out}}
\newcommand{\hyp}{\mathcal{H}_{Q,T}}
\newcommand{\Min}{\mathsf{Min}}

\newcommand{\perm}{\mathfrak{S}}
\newcommand{\fE}{\mathfrak{E}}
\newcommand{\fM}{\mathfrak{M}}

\newcommand{\un}{\mathbbmss{1}}

\newcommand{\Res}{\mathsf{Res}}
\newcommand{\Atoms}{\mathsf{Atoms}}
\newcommand{\PAtoms}{\mathsf{PAtoms}}

\def\cal#1{\expandafter\def\csname m#1\endcsname{\mathcal{#1}}}
\def\allcal#1{\ifx#1\allcal\else\cal#1\expandafter\allcal\fi}
\def\bold#1{\expandafter\def\csname b#1\endcsname{\mathbb{#1}}}
\def\allbold#1{\ifx#1\allbold\else\bold#1\expandafter\allbold\fi}
\allcal ABCDEFGHIJKLMNOPQRSTUVWXYZ\allcal
\allbold ABCDEFGHIJKLMNOPQRSTUVWXYZ\allbold

\begin{document}
\fontsize{12}{14}\selectfont

%\pagenumbering{gobble}
\hypersetup{pageanchor=false}
\begin{center}
    \rule{\linewidth}{0.3mm} \\[0.4cm]
    {\huge \bfseries Nominal Brzozowski -- Notes \& Proofs}\\
    \rule{\linewidth}{0.3mm} \\[0.4cm]
    \begin{center} \Large
        Killian \textsc{Barbé}\\\vspace{1em}
    \end{center}
\end{center}

\setlength{\parindent}{0pt}

Unless stated otherwise, we consider non-deterministic orbit-finite nominal automata \textbf{with} guessing in these notes.

\section{Atomic Nominal Automata}

\subsection{Definitions}

Throughout this part, we consider some orbit-finite non-deterministic nominal automaton $\mN=(Q,\delta,I,F)$ over $\Sigma=\bA$, recognising a language $\mL$.

\begin{definition}[Right language]

Given a state $q\in Q$ of $\mN$, we call right language of $q$ the language $$\mL_{q,F}^\mN=\{w\in\Sigma^*\mid\delta(q,w)\cap F\neq\varnothing\}$$

We analogously define the left language $\mL_{\mI,q}^{\mN}$ of q.

\end{definition}

\begin{note*}

The set of right languages $\{\mL_{q,F}^\mN\mid q\in Q\}$ of $\mN$ is orbit-finite since the function which maps states to their right language is equivariant.

\end{note*}

\begin{definition}[Residual languages]

We define the residual languages of $\mL$ to be those of the form $w^{-1}\mL$ for some word $w$. A non-deterministic nominal automaton will be called residual when each of its right languages is a residual language. We denote by $\Res(\mL)$ the set of residual languages of $\mL$.

\end{definition}

\begin{note*}

It follows from Myhill-Nerode that $\Res(\mL)$ is orbit-finite if and only if $\mL$ is recognised by an orbit-finite deterministic nominal automaton.

\end{note*}

\begin{definition}[Atoms of a language]

An atom of $\mL$ is a non-empty intersection of all (possibly complemented) residual languages, i.e. a language of the form $$A=\bigcap_{\mK\in\Res(\mL)}\widetilde{\mK}$$ (the tilde represents possible complementation). We write $\Atoms(\mL)$ the set of atoms of $\mL$.

\end{definition}

\begin{definition}[Partial atoms of an automaton]

A partial atom of $\mN$ is a non-empty intersection of all (possibly complemented) right languages of $\mN$, i.e. a language of the form $$X=\bigcap_{q\in Q}\widetilde{\mL_{q,F}^\mN}$$

We write $\PAtoms(\mN)$ the set of partial atoms of $\mN$, and denote by $$\varphi:\PAtoms(\mN)\to \mP(Q)$$ the function which maps a partial atom of $\mN$ to the set of states whose right languages appear uncomplemented in $X$.

\end{definition}

\begin{note*}

Each right language is a union of partial atoms (it is the union of all partial atoms in which it appears uncomplemented).

\end{note*}

\begin{definition}[Atomic states \& automata]

We call a state $q$ of $\mN$ atomic if its right language is a union of atoms. Moreover, we say that $\mN$ is atomic if all of its states are.

\end{definition}

\section{Reverse-determinisation and atomicity}\label{sec2}

We stick with the notations of the previous section, and denote by $D$ the operation of determinisation (note that it does not necessarily output an orbit-finite automaton) and by $R$ the operation of reversal. We also use the notation $\mL^R$ to denote the language recognised by $R(\mN)$. This section aims at proving the following result:

\begin{theorem}\label{thm-revdetmin}

A nominal automaton $\mN$ is atomic if and only if $D(R(\mN))$ is an observable automaton recognising $\mL^R$.

\end{theorem}

Note that this does not necessarily mean that the $D(R(\mN))$ is orbit-finite, since non-deterministic orbit-finite nominal automata recognise strictly more languages than deterministic orbit-finite ones. Here \textit{observable} means that no two reachable states have the same right language. We denote by $(\mQ,\Delta,\mI,\mF)$ the components of the automaton $D(R(\mN))$.

Let us state a very straightforward lemma before starting the proof of the theorem:

\begin{lemma}

For any partial atom $X$ of $\mN$, we have: $$\mL_{\mI,\varphi(X)}^{D(R(\mN))}=X^R$$

\end{lemma}

\begin{proof}

This is a direct consequence of the definition of atoms:

\begin{align*}\mL_{\mI,\varphi(X)}^{D(R(\mN))}&=\{w\in\Sigma^*\mid\Delta(\mI,w)=\varphi(X)\}\\
&=\{w\in\Sigma^*\mid\delta(q,w^R)\cap F\neq\varnothing\iff q\in\varphi(X)\}\\
&=\{w^R\in\Sigma^*\mid w\in\mL_{q,F}^\mN\iff q\in\varphi(X)\}\\
&=X^R
\end{align*}

\end{proof}

\begin{proof}[Proof of Theorem \ref{thm-revdetmin}] This is an attempt at proving this theorem as directly as possible (in Brzozowski's paper, the proof relies on an absurd amount of lemmas/theorems).

\begin{itemize}

\item Suppose $D(R(\mN))$ is observable and consider some partial atom $X$ of $\mN$. Recall that the left language of $\varphi(X)$ in $D(R(\mN))$ is exactly $X^R$. Since we have defined partial atoms to be non-empty, this directly implies that $\varphi(X)$ is reachable in $D(R(\mN))$. 

Let us show that $X$ corresponds exactly to the following atom of $\mL$:

$$A = \bigcap_{u\in\Sigma^*}\begin{cases}\overline{u^{-1}\mL} & \text{if }u^R\not\in\mL_{\varphi(X),\mF}^{D(R(\mN))}\\u^{-1}\mL & \text{otherwise}\end{cases}$$

\begin{itemize}

\item Suppose that $w\in A$. Note that we have $w\in u^{-1}\mL$ (i.e. $uw\in\mL$) if and only if $$u^R\in\{w\mid\Delta(\varphi(X),w)\in\mF\}=\{w\mid\delta(I,w^R)\cap\varphi(X)\neq\varnothing\}$$ 
We will now use the observability of $D(R(\mN))$ to prove the equality $\Delta(\mI,w^R)=\varphi(X)$ on states of this automaton. Let us show that these states have the same right language:
\begin{align*}
\Delta(\Delta(I,w^R),u)\in\mF&\iff\Delta(I,w^Ru)\in\mF\\
&\iff w^Ru\in\mL^R\\
&\iff u^Rw\in\mL\\
&\iff u\in\{w\mid\Delta(\varphi(X),w)\in\mF\}\\
&\iff\Delta(\varphi(X),u)\in\mF
\end{align*}

Since $D(R(\mN))$ is observable and both states involved are reachable, this shows that $\Delta(\mI,w^R)=\varphi(X)$. This gives directly $w^R\in \mL_{\mI,\varphi(X)}^{D(R(\mN))}=X^R$, so we have $w\in X$.

\item Suppose that $w\in X$. Then $w^R\in X^R$ so $\Delta(\mI,w^R)=\varphi(X)$. It follows directly that $uw\in\mL$ (i.e. $w^Ru^R\in\mL^R$) if and only if $u^R\in\mL_{\varphi(X),\mF}^{D(R(\mN))}$.

\end{itemize}

Hence $A=X$, so $X$ is an atom of $\mL$. Since every right language of $\mN$ is a union of partial atoms, it is a union of atoms so $\mN$ is atomic.

\item Now suppose $\mN$ is atomic and consider two different reachable states $S_1,S_2$ of $D(R(\mN))$, reached respectively by some words $u_1$ and $u_2$. We will show that there exists some word $w\in\mL_{S_1,\mF}^{D(R(\mN))}$ such that $w\not\in\mL_{S_2,\mF}^{D(R(\mN))}$.

Since $S_2$ is reached by $u_2$, we have $u_2^R\in\mL_{q_2,F}^\mN$ for exactly those $q_2$ which are in $S_2$. Let us fix such a state $q_2\in S_2\setminus S_1$ (or conversely if $S_2\setminus S_1=\varnothing$). If we had $u_1^R\in \mL_{q_2,F}^\mN$ then we would have $u_1\in\mL_{\mI,q_2}^{R(\mN)}$ so $q_2\in S_1$ which is impossible so, by atomicity of $\mN$, there exists some atom $A$ such that $u_2^R\in A\subseteq \mL_{q_2,F}^\mN$ sand $u_1^R\not\in A$. We then have $$\{w\in\Sigma^*\mid wu_1^R\in\mL\}\neq\{w\in\Sigma^*\mid wu_2^R\in\mL\}$$ Choosing some $w$ such that $wu_1^R\in\mL$ and $wu_2^R\not\in\mL$ (or conversely), we have $u_1w^R\in\mL^R$ and $u_2w^R\not\in\mL^R$ which implies that $w^R\in\mL_{S_1,\mF}^{D(R(\mN))}$ and $w^R\not\in\mL_{S_2,\mF}^{D(R(\mN))}$.

This shows that $D(R(\mN))$ is observable, which concludes our proof.

\end{itemize}

\end{proof}

\section{Expressivity of atomic nominal automata}

This section aims at solving the following question:

\begin{theorem}

Any non-deterministic orbit-finite nominal automaton is equivalent to an atomic orbit-finite nominal automaton.

\end{theorem}

Using the results of section \ref{sec2}, this boils down to showing that for any language $\mL$ recognised by some non-deterministic orbit-finite nominal automaton there exists a specific automaton which, when determinised, yields an observable (not necessarily orbit-finite) nominal automaton recognising $\mL$. I am not sure if this will be useful.

An other possibility is to prove the Venn result on the spanish algorithm and use the output of the algorithm as proof.

\bibliography{biblio}

\end{document}
